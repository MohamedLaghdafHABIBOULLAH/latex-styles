\documentclass{amsart}

\usepackage{amspaper}

% A draft watermark is nice while the paper is in preparation
% but it interferes with syncTeX
\usepackage{draftwatermark}
\SetWatermarkScale{8}
\SetWatermarkLightness{.97}

% For the Cahier du GERAD.
%\renewcommand{\year}{2020}     % If you don't want the current year.
\newcommand{\cahiernumber}{00}  % Insert your Cahier du GERAD number.

% For debugging.
%\usepackage{showframe}

% Meta-information for the PDF file generated.
\pdfinfo{/Author (Dominique Orban)
         /Title (Insert Title Here)
         /Keywords (keyword1, keyword2, keyword3)}

\begin{document}

\linenumbers

\title[Short Title]{%
  Long Title
}

%    author one information
\author[A. One]{Author One}
\address{%
  Some Lab, Some Univeristy, Some Place
}
\email{\mailto{author.one@someplace.edu}}
\thanks{}

%    author two information
\author[D. Orban]{Dominique Orban}
\address{%
  GERAD and
  Mathematics and Industrial Engineering Department \\
  \'Ecole Polytechnique, Montr\'eal, Canada
}
\urladdr{\http{www.gerad.ca/~orban}}
\email{\mailto{dominique.orban@gerad.ca}}
\thanks{Research partially supported by NSERC Discovery Grant 299010-04}

% see https://mathscinet.ams.org/msc/msc2010.html
\subjclass[2010]{
  15A06,  % Linear equations
  65F10,  % Iterative methods for linear systems
  65F20,  % Overdetermined systems, pseudoinverses
  65F22,  % Ill-posedness, regularization
  65F25,  % Orthogonalization
  65F35,  % Matrix norms, conditioning, scaling
  65F50,  % Sparse matrices
  93E24   % Least squares and related methods
  90C06,  % Large-scale problems
  90C20,  % Quadratic programming
  90C30,  % Nonlinear programming
  90C51,  % Interior-point methods
  90C53,  % Methods of quasi-Newton type
  90C55   % Methods of successive quadratic programming type
}

\keywords{%
  Keyword, keyword, keyword.
}

\date{\today}

\begin{abstract}
  This is a ground-breaking paper.
\end{abstract}

\maketitle

\pagestyle{myheadings}

\tableofcontents
\listoftodos\relax   % Must appear after toc.

\section{Introduction}

Consider
\begin{equation}
  \label{eq:nlp}
  \minimize{x \in  \R^n} \ f(x)
\end{equation}

\subsection*{Related Work}

\subsection*{Notation}


\section{Main Stuff}

\section{Conclusion}

\bibliographystyle{abbrvnat}
\bibliography{abbrv,\jobname}

\end{document}
